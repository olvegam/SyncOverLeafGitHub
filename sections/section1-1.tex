\section{First \texttt{import}ed section}

Below is a simple 3d plot. The image file is in the folder \texttt{img}, included with \texttt{subimport}

\begin{figure}[h]
\centering
\subimport{img/}{plot1.tex}
\caption{Caption}
\label{fig:my_label}
\end{figure}

The discipline of Business Process Management - BPM, was born as a promise to improve the way things were being done in organizations, to make them more efficient. Nowadays, that promise continues to be fulfilled in the day-to-day running of thousands of companies, since it has developed strategies and systems that allow for the integration of human and automatic activities in the management of complex business processes. Naturally, this entails an interconnection of different information systems and data sources, which serves to streamline a wide variety of activities in a business.  Above all, BPM facilitates the integration of new technological advances into them \cite{Abbott2021,Gazova2022,Harmon2019,Jimenez-Ramirez2021,Plattfaut2022,Ribeiro2022,Rinderle-Ma2021,Roglinger2021}. 

The implementation of BPM strategies in organizations was initially done through process-aware information systems, such as CRM, ERP, SCM and SLP. Then, the need to manage end-to-end processes and integrate different information systems gave rise to Workflow Management Systems, which focused on process modeling and execution. Later, Business Process Management Systems - BPMSs emerged as quite complete and versatile platforms with process monitoring and mining capabilities, which, furthermore, offered a greater ability to support SOA architectures, as well as integration with third-party applications and social networks \cite{Dumas2018}. 

A well-built BPMS might be able to manage different business processes automatically, e.g., design, analysis, execution, and monitoring\cite{Dumas2018}. All BPMSs, however, require a process model that may be used as an input.  BPMN is the current graphical language used to code this kind of processes\cite{OMG2015}; The main structural components of this language are actors, activities, events, flows and flow-direction changes (gateways). Executing any business process requires an integration of different information systems and data sources into the BPMS, which may be done through events, activities or gateways. Those information systems and data sources might either be based on JavaEE, .NET or web frameworks like Django. After such integrations are properly configured, the BPMS allows its users to monitor all the instances of a business process.     

The power and versatility of BPMSs makes them virtually indispensable nowadays for any BPM initiative, both in research and in day-to-day business operations. However,  the great variety of available BPMSs  makes it difficult to choose the most appropriate one. Moreover, this choice will likely vary according to the economic capacity and the specific purposes of each BPM initiative. 

This research....
%e present study is the first stage of a forthcoming research that aims to create a traceability system, from BPMN models to execution links.  Since our purpose is to have the greatest impact in the BPM software community through an accessible framework, we had to choose a BPMS that did not hinder this objective. Thus, we had to look for a free and open-source BPMS that, furthermore, had a sizable community around it. This paper presents  the criteria that we used to choose the BPM platform that we will use from now on for our research.   

In the %\nameref{sec:Backg} 
section of this study, we summarize the main concepts related to BPM 
platforms (BPM strategy, BPMN graphic language, and information-system integration), and describe the relationship between BPM and the software that supports the operation of an enterprise. With these concepts in place, we explain the 
%\nameref{sec:Methodology} 
for selecting the BPM platforms that we compared and the evaluating criteria we used to select the most appropriate one for our purposes. In 
%\nameref{sec:Results} 
we summarize the scores obtained by each of the BPM platforms evaluated and present the  one with the greatest score. Then, in 
%\nameref{sec:Threats}
, we describe some peculiarities of the selection process that could be construed as weaknesses, but which, in light of the results obtained, might not be so relevant. Afterwards, in 
%\nameref{sec:RelWork}
, we present other BPM platform comparisons made by other teams and explain how this work differs from them. Finally, in 
%\nameref{sec:Conclusions}
, we summarize our findings and reinstate the reasons of our choice. 

%\Blindtext
